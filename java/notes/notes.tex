\documentclass[utf8]{ctexart}
\usepackage{tabularx}
\begin{document}
    \section{java笔记}
    线程关闭,不可调用stop方法,因为可能有资源没有释放掉,可以设置一个变量通知线程,当值为-1时,return run方法(合理关闭)
    线程分类:daemon守护线程,用户线程

    iterator迭代器,
    失败:集合中数据被改了,数据肯定就不对了,我们就称为失败
        快速失败:遍历数据是集合本身,会抛出异常;
        安全失败:遍历的是集合数据的副本,不会出现异常;
        如果API不特别说明,默认安全失败;

    map(interface):mapping
        一个键只能对应一个值

    hashcode:int值
            值会因对象、值变化,而分布的比较均匀;相同属性也不建议值一样
    哈希表:对象数组+链表
        依据对象的hashcode值来和数组长度取余运算,得到的数字作为下标,放入数组,这样查找快
        余数相等就用链表存,
        数组中的每一个下标,叫哈希桶
        哈希桶的长度大于8时,转换成红黑树
        当哈希桶中的数据量减少到6时,从红黑树转换成链表
            问:如果哈希桶中数据为7个,一定会从红黑树转为链表么?
                答:不一定,原来到7,可能还是链表结构,不用转
    hashmap: 
        影响性能的两个参数:
            1.初始(桶)数量:16(扩容为2倍原长度)!注意,一旦桶的数量更改了(即下标范围变更了),需要重新取余计算
            2.散列因子(加载因子):0.75(有75\%的桶都装有数据了,进行扩容即散列(通过重建内部数据结构));该参数反映在存储空间和查找效率上;
        有数组,就算哈希表了,不一定非要有链表存在。
        存储值时,先依据键计算哈希值,确定存放位置;
        源码中用(数组长度-1)和哈希值与运算就等于哈希值取余长度
        里面的key值尤其是自定义类型,不要乱改值,不然就找不到了,
        确认两个键相等,需要满足哈希值相等,还有满足equals。
    

    list,map,set接口提供了固定长度的集合,of()重载方法

    HashMap:不保证存储有序
    TreeMap:不保证存储有序,自动排序
    LinkedHashMap:保证存储有序,通过使用双向链表和哈希表来完成,既保证顺序,也保证查找高效


    set不保证顺序,属性更改会导致不同位置


    需要创建一个id,setting中,复制,已完成
    编译速度太快,更改,已完成
\end{document}

