\documentclass[12pt]{ctexart}
\usepackage{tabularx}
\usepackage{color}
\usepackage[colorlinks,linkcolor=blue]{hyperref}
\begin{document}
%\newcommand{\myfont}{\textit{\textbf{\textsf{Fancy Text}}}}
\title{Java}
\author{黄浩}
\date{2021-2-8 19:27}
\maketitle
\tableofcontents
% \part{}
% %\chapter{嗯?}
% \section{}
% \subsection{}
% \subsubsection{}
% \paragraph{}
% \subparagraph{}
\part{Java基础部分}
\begin{abstract}
    java基础笔记
\end{abstract}
\section{线程}
线程关闭,不可调用stop方法,因为可能有资源没有释放掉,可以设置一个变量通知线程,当值为-1时,return run方法(合理关闭)
\paragraph{}{\kaishu 线程分类:daemon守护线程,用户线程}
\section{集合}
\subsection{迭代器}
\subsubsection{iterator迭代器}
失败:集合中数据被改了,数据肯定就不对了,我们就称为失败;\paragraph{}
$\ \ $快速失败:遍历数据是集合本身,会抛出异常;\paragraph{}
\quad 安全失败:遍历的是集合数据的副本,不会出现异常;\paragraph{}
\quad 如果API不特别说明,默认安全失败;\paragraph{}
%剩余空格方式
% \ \ 如果API不特别说明,默认安全失败;\paragraph{}
% ~如果API不特别说明,默认安全失败;\paragraph{}
% ~~~~~~如果API不特别说明,默认安全失败;
%\qquad 这里缩进会有\quad两倍
\subsection{map集合}
map(interface):mapping
一个键只能对应一个值
\subsection{哈希表及其相关知识}
\subsubsection{hashcode}
hashcode:int类型值\paragraph{}
值会因对象、值变化,而分布的比较均匀;相同属性也不建议值一样
\subsubsection{哈希表}
哈希表:对象数组+链表(java中)
依据对象的hashcode值来和数组长度取余运算,得到的数字作为下标,放入数组,这样查找快
余数相等就用链表存,
数组中的每一个下标,叫哈希桶。哈希桶的长度大于8时,转换成红黑树
当哈希桶中的数据量减少到6时,从红黑树转换成链表
\\
问:如果哈希桶中数据为7个,一定会从红黑树转为链表么?
答:不一定,原来到7,可能还是链表结构,不用转
\subsubsection{hashmap}
hashmap:
\paragraph{}
影响性能的两个参数:\subparagraph{}
1.初始(桶)数量:16(扩容为2倍原长度)!注意,一旦桶的数量更改了(即下标范围变更了),\textcolor{magenta}{需要重新取余计算} \subparagraph{}
2.散列因子(加载因子):0.75(有75\%的桶都装有数据了,进行扩容即散列(通过重建内部数据结构));该参数反映在存储空间和查找效率上;
有数组,就算哈希表了,不一定非要有链表存在。
存储值时,先依据键计算哈希值,确定存放位置;
源码中用(数组长度-1)和哈希值与运算就等于哈希值取余长度
里面的key值尤其是自定义类型,不要乱改值,不然就找不到了,
确认两个键相等,需要满足哈希值相等,还有满足equals。


list,map,set接口提供了固定长度的集合,of()重载方法

HashMap:不保证存储有序
TreeMap:不保证存储有序,自动排序
LinkedHashMap:保证存储有序,通过使用双向链表和哈希表来完成,既保证顺序,也保证查找高效

\subsection{set}
set不保证顺序,属性更改会导致不同位置

\section{备忘}
需要创建一个id,setting中,复制,已完成\paragraph{}
编译速度太快,更改,已完成\paragraph{}
VSCode 默认 单击文件是预览, 双击文件是用一个新的Tab打开。如果想关掉预览模式,单击打开文件,在设置里加入
\href{https://blog.csdn.net/qq_41865652/article/details/107024390}{参考文档}\\
或者更改JSON文件"workbench.editor.enablePreview": false\ 这样单击文件就是直接用新Tab打开了。
\end{document}

